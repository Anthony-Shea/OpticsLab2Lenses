\documentclass[12pt]{article}

\usepackage[margin=1in]{geometry}
\usepackage{setspace}
\usepackage{graphicx}
\usepackage{amsmath}
\setstretch{1.2}

\title{Lab 2: Lenses}
\author{
    Anthony Shea \\[0.3em]
    Course Number: PHYS 353 \\
    Lab Section: 3 \\[0.3em]
    Institution: Colorado State University \\[0.3em]
    Lab Partner: Jesse Kiedrowski \\[0.3em]
    Lab Partner: Ellen Raad
}
\date{\today}

\begin{document}

\maketitle

\section{Objective}



\section{Theory}

Thin lens equation:

\begin{equation}
\frac{1}{f} = \frac{1}{d_o} + \frac{1}{d_i}
\end{equation}

This was from Physics of Light and Optics by Peatross and Ware equation 7.4.3 \cite[p.~230]{peatross}.

However, our lab manual uses the following version:

\begin{equation}
\frac{1}{s} + \frac{1}{s'} = \frac{1}{f}
\end{equation}

Below is an picture of the lab manual showing the variables used in the thin lens equation.


\begin{figure}
    \centering
    \includegraphics[width=0.50\textwidth, angle=0]{ThinLens.JPG}
    \caption{This image shows the variables used in the thin lens equation.}
    \label{fig:BA2_Measurements_2}
\end{figure}

Uncertainty in a Function of Several Variables

\begin{equation}
\delta q = \sqrt{(\frac{\partial q}{\partial x} \delta x)^2 + ... + (\frac{\partial q}{\partial z} \delta z)^2 }
\end{equation}

This was from An Introduction to Error Analysis by John R. Taylor equation 3.47 \cite[p.~75]{taylor}.



\section{Experimental Procedure}



\section{Experimental Results and Analysis}



\section{Conclusions}



\bibliographystyle{plain}
\bibliography{biblio}
\end{document}